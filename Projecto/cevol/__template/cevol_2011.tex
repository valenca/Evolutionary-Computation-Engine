\documentclass[12pt,portuguese,a4paper]{article}
\usepackage[T1]{fontenc}
\usepackage[scaled]{luximono}
\usepackage{amsmath}
\usepackage{babel}
\usepackage{graphicx}
\usepackage{epstopdf}
\DeclareGraphicsRule{.tif}{png}{.png}{`convert #1 `dirname #1`/`basename #1 .tif`.png}

%uncomment line below for Windows plarforms
%\usepackage[latin1]{inputenc}

% comment line below for Windows platforms
\usepackage[applemac]{inputenc}

\usepackage{natbib}
\usepackage{color}
\usepackage{colortbl}
\usepackage[colorlinks,urlcolor=blue]{hyperref}
\usepackage{fancybox,calc}
\usepackage{alltt}
\usepackage[Lenny]{fncychap}
\usepackage{makeidx}
\usepackage{fancyvrb}
\usepackage{algorithm2e}


%%% PACKAGES
\usepackage{booktabs} % for much better looking tables
\usepackage{array} % for better arrays (eg matrices) in maths
\usepackage{paralist} % very flexible & customisable lists (eg. enumerate/itemize, etc.)
\usepackage{verbatim} % adds environment for commenting out blocks of text & for better verbatim
\usepackage{subfigure} % make it possible to include more than one captioned figure/table in a single float
% These packages are all incorporated in the memoir class to one degree or another...




%%% Para o c�digo
\usepackage{listings}
\lstloadlanguages{Python,C,Java,Prolog,Lisp}

\lstset{backgroundcolor=\color{yellow},language=Python,captionpos=b,showstringspaces=false,basicstyle=\ttfamily,commentstyle=\color{blue},tabsize=2,breaklines=true}

%% E mais umas coisas
\usepackage{pifont}
\usepackage{lettrine}
\usepackage{soul}
\usepackage{epigraph}
%\usepackage{chicago}

% See the ``Article customise'' template for come common customisations

\title{\Large{\textbf{T�tulo }}}
\author{-- Autor-- \\ Computa\c c�o Evolucion�ria\\ MEI -2013/2014 \\ Universidade de Coimbra \\ -- email --}
\date{\today} % delete this line to display the current date


\renewcommand\lstlistingname{Listagem}


%%% BEGIN DOCUMENT
\begin{document}




\maketitle


\begin{abstract}

O resumo �  o local onde se faz a s�ntese do documento.
Este texto pretende ilustrar o que se pretende com a \textbf{ficha de leitura} relativa ao texto te�rico que contam para a avalia��o na cadeira. S�o descritas as suas v�rias sec��es e o que se pretende com cada uma delas.  Este documento deve ter um \textbf{m�ximo de tr�s p�ginas de texto}, n�o sendo obrigat�rio a sua escrita em \LaTeX{}. A inclus�o de figuras e/ou tabelas n�o  interfere com este limite.Pode ser escrito em portugu�s ou em ingl�s.
\end{abstract}

\section{What}

Aqui se descreve a refer�ncia do trabalho e se d� o contexto, isto �,  a �rea e o \textbf{tema} do artigo (do ponto de vista do autor).

\section{Why}

Porque � que o \textbf{autor} acha a quest�o relevante? Porque � que \textbf{n�s} achamos a quest�o relevante? Aqui se colocam as respostas a estas duas quest�es.

\section{Contribution(s)}

O que \textbf{prop�e} o autor? Caso exista uma an�lise experimental, quais os \textbf{resultados} que suportam as conclus�es do autor?


\section{Like/ Don't Like}

Esta � uma parte relevante do trabalho, para efeitos de avalia��o. Identifique os aspectos \textbf{positivos} e \textbf{negativos} da proposta. Que \textbf{voc�} alternativas sugere? As ideias do autor podem ser adaptadas a outras quest�es?\\

\noindent {\Large \textbf{Bibliografia}}\\

Coloque nesta parte final a bibliografia adicional que eventualmente leu/ajudou a perceber a alisar o trabalho.




\bibliographystyle{plain}
% retirar coment�rio na pr�xima linha para incluir bibliografia
%\bibliography{??}

\end{document}